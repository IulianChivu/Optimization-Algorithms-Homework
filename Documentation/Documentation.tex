\documentclass[a4paper]{article}
\usepackage{fullpage}
\title{\huge Tehnici de optimizare in programare
\\Raport tema de curs
}
\author{Chisalescu Bogdan  si  Chivu Gheorghe-Iulian
\\ Grupa 432A
}
\date{}

\begin{document}
\maketitle
\vphantom{} \\ \\ \\ \\ \\
\section{Introducere}
\Large 
Algoritmii genetici sunt o subclasa a algoritmilor evolutivi ce se pot aplica in orice problema de optimizare. Acestia sunt algoritmi matematici inspirati din teoria evolutiei speciilor ce se bazeaza pe un set de caracteristici. Acest set include existenta unei populatii (indivizii reprezentand posibile solutii), o modalitate de a genera noi indivizi in populatie (prin recombinarea a doi indivizi deja existenti si/sau mutatia unui individ) si o modalitate ce simuleaza selectia naturala (un algoritm prin care se decide ce indivizi raman in populatie si modul in care se modifica dimensiunea populatiei).
\newline
Prezenta lucrare are ca scop implementarea unui algoritm genetic specific apartinand unei lucrari de studiu de cercetare. Lucrarea de cercetare se intituleaza "An Effective Real-Parameter Genetic Algorithm with Parent Centric Normal Crossover for Multimodal Optimisation" si este realizata de Pedro J. Ballester si Jonathan N. Carter in anul 2004. 
\newline
Urmatoarele sectiuni ce vor fi descrise sunt o imagine a modului in care am reusit sa intelegem notiunile teoretice din spatele acestui studiu, a modului in care am reusit sa implementam algoritmul si in care am interpretat rezultatele obtinute; si nu o reprezentare exacta a studiului, astfel am incercat pe cat posibil sa reproducem algoritmul si performantele sale.
\newpage

\section{Descrierea Algoritmului}

Algoritmul descris de studiul de cercetare se intituleaza "Scaled Probabilistic Crowding Genetic Algorithm with Parent Centric Normal crossover" la care ne vom referi pe tot parcursul lucrarii ca SPC-PNX. 

\end{document}